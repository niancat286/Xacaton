\documentclass[]{article}
\usepackage{lmodern}
\usepackage{amssymb,amsmath}
\usepackage{ifxetex,ifluatex}


\usepackage[utf8]{inputenc}
\usepackage[english,russian,ukrainian]{babel}

\usepackage{fixltx2e} % provides \textsubscript
\ifnum 0\ifxetex 1\fi\ifluatex 1\fi=0 % if pdftex
  \usepackage[T1]{fontenc}
  \usepackage[utf8]{inputenc}
\else % if luatex or xelatex
  \ifxetex
    \usepackage{mathspec}
  \else
    \usepackage{fontspec}
  \fi
  \defaultfontfeatures{Ligatures=TeX,Scale=MatchLowercase}
\fi
% use upquote if available, for straight quotes in verbatim environments
\IfFileExists{upquote.sty}{\usepackage{upquote}}{}
% use microtype if available
\IfFileExists{microtype.sty}{%
\usepackage{microtype}
\UseMicrotypeSet[protrusion]{basicmath} % disable protrusion for tt fonts
}{}
\usepackage[unicode=true]{hyperref}
\hypersetup{
            pdfborder={0 0 0},
            breaklinks=true}
\urlstyle{same}  % don't use monospace font for urls
\usepackage{graphicx,grffile}
\makeatletter
\def\maxwidth{\ifdim\Gin@nat@width>\linewidth\linewidth\else\Gin@nat@width\fi}
\def\maxheight{\ifdim\Gin@nat@height>\textheight\textheight\else\Gin@nat@height\fi}
\makeatother
% Scale images if necessary, so that they will not overflow the page
% margins by default, and it is still possible to overwrite the defaults
% using explicit options in \includegraphics[width, height, ...]{}
\setkeys{Gin}{width=\maxwidth,height=\maxheight,keepaspectratio}
\IfFileExists{parskip.sty}{%
\usepackage{parskip}
}{% else
\setlength{\parindent}{0pt}
\setlength{\parskip}{6pt plus 2pt minus 1pt}
}
\setlength{\emergencystretch}{3em}  % prevent overfull lines
\providecommand{\tightlist}{%
  \setlength{\itemsep}{0pt}\setlength{\parskip}{0pt}}
\setcounter{secnumdepth}{0}
% Redefines (sub)paragraphs to behave more like sections
\ifx\paragraph\undefined\else
\let\oldparagraph\paragraph
\renewcommand{\paragraph}[1]{\oldparagraph{#1}\mbox{}}
\fi
\ifx\subparagraph\undefined\else
\let\oldsubparagraph\subparagraph
\renewcommand{\subparagraph}[1]{\oldsubparagraph{#1}\mbox{}}
\fi

\date{}

\usepackage{multicol}

\usepackage{enumitem}
\makeatletter
\newcommand{\xslalph}[1]{\expandafter\@xslalph\csname c@#1\endcsname}
\newcommand{\@xslalph}[1]{%
    \ifcase#1\or а\or б\or в\or г\or д\or e\or є\or ж\or з\or i%
    \or й\or к\or л\or м\or н\or о\or п\or р\or с\or т%
    \or у\or ф\or х\or ц\or ч\or ш\or ю\or я\or аа\or бб\or вв %
    \else\@ctrerr\fi%
}
\AddEnumerateCounter{\xslalph}{\@xslalph}{m}
\makeatother


\begin{document}

Василь Пупкін проходить тестове завдання при прийомі на роботу в IT-компанію. 
Йому необхідно реалізувати систему роботи з багатокутниками, яка відповідає наступним вимогам.
Але нажаль, Василь невміє користуватися Гітхабом, тому частина коду пропала, йому терміново (до 14-00)
потрібно все відновити.
При цьому існуючий код точно правильний, його міняти неможна, але файли майже всі трохи недописані до кінця...

Відомо, що тести та код були вже написані. Потрібно написати та протестувати всі пункти даних задач. 
Тобто, команда пише розв'язки та - відповідні тести.
На кожний пункт повинно буте >= 3 різних юніт тестів.   

Результат розв'язків даних задач повинен міститись в даній папці за назвою "Hakaton25_(номер команди)"
та запускатись даним мейкфайлом.
В заголовні файли можна додавати щось, але не можна видаляти чи змінювати їх.

Увага! Користуватись колекціями С++ не можна - задачі на "чистому" С!

В наступних задачах умови $Q(x)$ задаються предікатом, або 
булевою функцією, що передається через вказівник як аргумент.
\begin{enumerate}
\item 

Всі дані повинні разом вводяться у файл з клавіатури або з файлу input.dat, 
виводяться на екран та у файл output.txt.

Задача - реалізувати структуру даних для зберігання багатокутників, що зберігаются у файлі. 
Файл з багатокутником зберігається в текстовому файлі.

Формат файлу для множини багатокутників:
На початку - unsigned число $M$ - кількість багатокутників ($1 \le N \le 100000$). Далі
записано $M$ структур, які складаються з кількості вершин багатокутника та $N$ пар float чисел 
що є координатами вершин $(x,y)$.

Сформуйте файл багатокутників, в якому жоден елемент не повторюється.
Інтерфейс:

Визначити:
\begin{enumerate}[label=\xslalph*)]
\item процедуру введення та додавання багатокутника з консолі у файл (з перевіркою коректності даних);
\item процедуру введення та додавання всіх багатокутників з іншого файлу у вихідний файл;
\item процедуру виведення всіх багатокутників;
\item процедуру виведення багатокутника за даним індексом на консоль;
\item процедуру видалення багатокутника по індексу з файлу;
\item функцію, що дає відповідь, чи є даний багатокутник у нашому файлі (багатокутникі однакові, якщо всі їх відповідні сторони та кути - рівні);
\item функцію, що знаходить максимальний багатокутник за периметром;
\item функцію, що знаходить мінімальний за площею;
\item функцію, що знаходить кількість опуклих багатокутників у файлі;
\item функцію, що знаходить кількість багатокутників у файлі в середині яких є дана точка Р;
\item функцію, що за файлу A знаходить підмножину всіх таких її
багатокутників, для яких справедлива умова $Q(х)$, $x\in A$ та записує у файл В;

\end{enumerate}


Додаткова інформація в заголовних та кодових файлах.

\subsection*{Технічні вимоги}

\begin{itemize}
\item Мова програмування: C (заборонено використання колекцій C++)
\item Всі дані повинні вводитися з консолі або з файлу \texttt{input.dat}
\item Результати роботи програми повинні виводитися на екран та у файл \texttt{output.txt}
\item Кожна функція повинна мати щонайменше 3 юніт-тести
\item Код повинен компілюватися з наданим Makefile
\item Розв'язок розмістити в папці \texttt{Hakaton25\_<номер команди>}
\end{itemize}

\subsection*{Критерії оцінювання}

\begin{itemize}
\item Коректність реалізації всіх функцій 
\item Якість та покриття тестами 
\item Оптимізація роботи з пам'яттю 
\item Читабельність та структурування коду
\end{itemize}

\end{enumerate}

\end{document}
